\documentclass[a4paper,12pt]{article}

% Packages
\usepackage{graphicx} % For including images
\usepackage{hyperref} % For hyperlinks
\usepackage{amsmath}  % For mathematical expressions
\usepackage{amssymb}  % For additional math symbols
\usepackage{listings} % For code listings

% Title and Author
\title{Mathix: A C++ Computer Algebra System \\ \large Version 0.1.0}
\author{Sergio Rodriguez Freire}
\date{\today}

\begin{document}

% Title Page
\maketitle
\begin{center}
    \includegraphics[width=0.5\textwidth]{logo.png} % Resize the logo to fit
\end{center}

\tableofcontents
\newpage

% Introduction
\section{Introduction}
Mathix is a powerful and flexible computer algebra system written in C++. It allows users to perform symbolic computations, evaluate mathematical expressions, and define custom functions. This document serves as a user manual for Mathix, guiding users through its features and functionalities.

% Basic Syntax and Mathematical Expressions
\section{Basic Syntax and Mathematical Expressions}
Mathix currently supports simple algebraic operations, including addition (`+`), subtraction (`-`), multiplication (`*`), and division (`/`). Below are examples of how to use Mathix for basic computations.

\subsection{Using the Mathix CLI}
When you start Mathix, you will see the following prompt:
\begin{lstlisting}
Welcome to Mathix CLI!
Type 'exit' to quit.
\end{lstlisting}

You can then enter mathematical expressions, and Mathix will evaluate them. For example:

\begin{lstlisting}
> 1+2
= 3.000000
> 1+(3*4)
= 13.000000
> 4/7
= 0.571429
\end{lstlisting}

% Advanced Features (Proposed Chapters)
\section{Advanced Features}
This section will cover advanced features of Mathix. These chapters are placeholders for now and will be expanded in the future.

\subsection{Custom Functions}
Details on how to define and use custom functions.

\subsection{Expression Simplification}
Techniques for simplifying mathematical expressions.

\subsection{Differentiation and Integration}
Symbolic differentiation and integration capabilities.

\subsection{Matrix Operations}
Support for symbolic and numerical matrix operations.

\subsection{Extending Mathix}
How to extend Mathix with custom types and operations.

% Conclusion
\section{Conclusion}
Mathix is a versatile tool for symbolic computation and mathematical analysis. This user manual provides an overview of its basic and advanced features. For further details, refer to the source code or contact the development team.

\end{document}